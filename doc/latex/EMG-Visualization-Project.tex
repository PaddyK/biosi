% Generated by Sphinx.
\def\sphinxdocclass{report}
\documentclass[letterpaper,10pt,english]{sphinxmanual}
\usepackage[utf8]{inputenc}
\DeclareUnicodeCharacter{00A0}{\nobreakspace}
\usepackage{cmap}
\usepackage[T1]{fontenc}
\usepackage{babel}
\usepackage{times}
\usepackage[Bjarne]{fncychap}
\usepackage{longtable}
\usepackage{sphinx}
\usepackage{multirow}

\addto\captionsenglish{\renewcommand{\figurename}{Fig. }}
\addto\captionsenglish{\renewcommand{\tablename}{Table }}
\floatname{literal-block}{Listing }



\title{EMG-Visualization-Project Documentation}
\date{October 07, 2015}
\release{0}
\author{Patrick Kalmbach, Justin Bayer}
\newcommand{\sphinxlogo}{}
\renewcommand{\releasename}{Release}
\makeindex

\makeatletter
\def\PYG@reset{\let\PYG@it=\relax \let\PYG@bf=\relax%
    \let\PYG@ul=\relax \let\PYG@tc=\relax%
    \let\PYG@bc=\relax \let\PYG@ff=\relax}
\def\PYG@tok#1{\csname PYG@tok@#1\endcsname}
\def\PYG@toks#1+{\ifx\relax#1\empty\else%
    \PYG@tok{#1}\expandafter\PYG@toks\fi}
\def\PYG@do#1{\PYG@bc{\PYG@tc{\PYG@ul{%
    \PYG@it{\PYG@bf{\PYG@ff{#1}}}}}}}
\def\PYG#1#2{\PYG@reset\PYG@toks#1+\relax+\PYG@do{#2}}

\expandafter\def\csname PYG@tok@gd\endcsname{\def\PYG@tc##1{\textcolor[rgb]{0.63,0.00,0.00}{##1}}}
\expandafter\def\csname PYG@tok@gu\endcsname{\let\PYG@bf=\textbf\def\PYG@tc##1{\textcolor[rgb]{0.50,0.00,0.50}{##1}}}
\expandafter\def\csname PYG@tok@gt\endcsname{\def\PYG@tc##1{\textcolor[rgb]{0.00,0.27,0.87}{##1}}}
\expandafter\def\csname PYG@tok@gs\endcsname{\let\PYG@bf=\textbf}
\expandafter\def\csname PYG@tok@gr\endcsname{\def\PYG@tc##1{\textcolor[rgb]{1.00,0.00,0.00}{##1}}}
\expandafter\def\csname PYG@tok@cm\endcsname{\let\PYG@it=\textit\def\PYG@tc##1{\textcolor[rgb]{0.25,0.50,0.56}{##1}}}
\expandafter\def\csname PYG@tok@vg\endcsname{\def\PYG@tc##1{\textcolor[rgb]{0.73,0.38,0.84}{##1}}}
\expandafter\def\csname PYG@tok@m\endcsname{\def\PYG@tc##1{\textcolor[rgb]{0.13,0.50,0.31}{##1}}}
\expandafter\def\csname PYG@tok@mh\endcsname{\def\PYG@tc##1{\textcolor[rgb]{0.13,0.50,0.31}{##1}}}
\expandafter\def\csname PYG@tok@cs\endcsname{\def\PYG@tc##1{\textcolor[rgb]{0.25,0.50,0.56}{##1}}\def\PYG@bc##1{\setlength{\fboxsep}{0pt}\colorbox[rgb]{1.00,0.94,0.94}{\strut ##1}}}
\expandafter\def\csname PYG@tok@ge\endcsname{\let\PYG@it=\textit}
\expandafter\def\csname PYG@tok@vc\endcsname{\def\PYG@tc##1{\textcolor[rgb]{0.73,0.38,0.84}{##1}}}
\expandafter\def\csname PYG@tok@il\endcsname{\def\PYG@tc##1{\textcolor[rgb]{0.13,0.50,0.31}{##1}}}
\expandafter\def\csname PYG@tok@go\endcsname{\def\PYG@tc##1{\textcolor[rgb]{0.20,0.20,0.20}{##1}}}
\expandafter\def\csname PYG@tok@cp\endcsname{\def\PYG@tc##1{\textcolor[rgb]{0.00,0.44,0.13}{##1}}}
\expandafter\def\csname PYG@tok@gi\endcsname{\def\PYG@tc##1{\textcolor[rgb]{0.00,0.63,0.00}{##1}}}
\expandafter\def\csname PYG@tok@gh\endcsname{\let\PYG@bf=\textbf\def\PYG@tc##1{\textcolor[rgb]{0.00,0.00,0.50}{##1}}}
\expandafter\def\csname PYG@tok@ni\endcsname{\let\PYG@bf=\textbf\def\PYG@tc##1{\textcolor[rgb]{0.84,0.33,0.22}{##1}}}
\expandafter\def\csname PYG@tok@nl\endcsname{\let\PYG@bf=\textbf\def\PYG@tc##1{\textcolor[rgb]{0.00,0.13,0.44}{##1}}}
\expandafter\def\csname PYG@tok@nn\endcsname{\let\PYG@bf=\textbf\def\PYG@tc##1{\textcolor[rgb]{0.05,0.52,0.71}{##1}}}
\expandafter\def\csname PYG@tok@no\endcsname{\def\PYG@tc##1{\textcolor[rgb]{0.38,0.68,0.84}{##1}}}
\expandafter\def\csname PYG@tok@na\endcsname{\def\PYG@tc##1{\textcolor[rgb]{0.25,0.44,0.63}{##1}}}
\expandafter\def\csname PYG@tok@nb\endcsname{\def\PYG@tc##1{\textcolor[rgb]{0.00,0.44,0.13}{##1}}}
\expandafter\def\csname PYG@tok@nc\endcsname{\let\PYG@bf=\textbf\def\PYG@tc##1{\textcolor[rgb]{0.05,0.52,0.71}{##1}}}
\expandafter\def\csname PYG@tok@nd\endcsname{\let\PYG@bf=\textbf\def\PYG@tc##1{\textcolor[rgb]{0.33,0.33,0.33}{##1}}}
\expandafter\def\csname PYG@tok@ne\endcsname{\def\PYG@tc##1{\textcolor[rgb]{0.00,0.44,0.13}{##1}}}
\expandafter\def\csname PYG@tok@nf\endcsname{\def\PYG@tc##1{\textcolor[rgb]{0.02,0.16,0.49}{##1}}}
\expandafter\def\csname PYG@tok@si\endcsname{\let\PYG@it=\textit\def\PYG@tc##1{\textcolor[rgb]{0.44,0.63,0.82}{##1}}}
\expandafter\def\csname PYG@tok@s2\endcsname{\def\PYG@tc##1{\textcolor[rgb]{0.25,0.44,0.63}{##1}}}
\expandafter\def\csname PYG@tok@vi\endcsname{\def\PYG@tc##1{\textcolor[rgb]{0.73,0.38,0.84}{##1}}}
\expandafter\def\csname PYG@tok@nt\endcsname{\let\PYG@bf=\textbf\def\PYG@tc##1{\textcolor[rgb]{0.02,0.16,0.45}{##1}}}
\expandafter\def\csname PYG@tok@nv\endcsname{\def\PYG@tc##1{\textcolor[rgb]{0.73,0.38,0.84}{##1}}}
\expandafter\def\csname PYG@tok@s1\endcsname{\def\PYG@tc##1{\textcolor[rgb]{0.25,0.44,0.63}{##1}}}
\expandafter\def\csname PYG@tok@gp\endcsname{\let\PYG@bf=\textbf\def\PYG@tc##1{\textcolor[rgb]{0.78,0.36,0.04}{##1}}}
\expandafter\def\csname PYG@tok@sh\endcsname{\def\PYG@tc##1{\textcolor[rgb]{0.25,0.44,0.63}{##1}}}
\expandafter\def\csname PYG@tok@ow\endcsname{\let\PYG@bf=\textbf\def\PYG@tc##1{\textcolor[rgb]{0.00,0.44,0.13}{##1}}}
\expandafter\def\csname PYG@tok@sx\endcsname{\def\PYG@tc##1{\textcolor[rgb]{0.78,0.36,0.04}{##1}}}
\expandafter\def\csname PYG@tok@bp\endcsname{\def\PYG@tc##1{\textcolor[rgb]{0.00,0.44,0.13}{##1}}}
\expandafter\def\csname PYG@tok@c1\endcsname{\let\PYG@it=\textit\def\PYG@tc##1{\textcolor[rgb]{0.25,0.50,0.56}{##1}}}
\expandafter\def\csname PYG@tok@kc\endcsname{\let\PYG@bf=\textbf\def\PYG@tc##1{\textcolor[rgb]{0.00,0.44,0.13}{##1}}}
\expandafter\def\csname PYG@tok@c\endcsname{\let\PYG@it=\textit\def\PYG@tc##1{\textcolor[rgb]{0.25,0.50,0.56}{##1}}}
\expandafter\def\csname PYG@tok@mf\endcsname{\def\PYG@tc##1{\textcolor[rgb]{0.13,0.50,0.31}{##1}}}
\expandafter\def\csname PYG@tok@err\endcsname{\def\PYG@bc##1{\setlength{\fboxsep}{0pt}\fcolorbox[rgb]{1.00,0.00,0.00}{1,1,1}{\strut ##1}}}
\expandafter\def\csname PYG@tok@kd\endcsname{\let\PYG@bf=\textbf\def\PYG@tc##1{\textcolor[rgb]{0.00,0.44,0.13}{##1}}}
\expandafter\def\csname PYG@tok@ss\endcsname{\def\PYG@tc##1{\textcolor[rgb]{0.32,0.47,0.09}{##1}}}
\expandafter\def\csname PYG@tok@sr\endcsname{\def\PYG@tc##1{\textcolor[rgb]{0.14,0.33,0.53}{##1}}}
\expandafter\def\csname PYG@tok@mo\endcsname{\def\PYG@tc##1{\textcolor[rgb]{0.13,0.50,0.31}{##1}}}
\expandafter\def\csname PYG@tok@mi\endcsname{\def\PYG@tc##1{\textcolor[rgb]{0.13,0.50,0.31}{##1}}}
\expandafter\def\csname PYG@tok@kn\endcsname{\let\PYG@bf=\textbf\def\PYG@tc##1{\textcolor[rgb]{0.00,0.44,0.13}{##1}}}
\expandafter\def\csname PYG@tok@o\endcsname{\def\PYG@tc##1{\textcolor[rgb]{0.40,0.40,0.40}{##1}}}
\expandafter\def\csname PYG@tok@kr\endcsname{\let\PYG@bf=\textbf\def\PYG@tc##1{\textcolor[rgb]{0.00,0.44,0.13}{##1}}}
\expandafter\def\csname PYG@tok@s\endcsname{\def\PYG@tc##1{\textcolor[rgb]{0.25,0.44,0.63}{##1}}}
\expandafter\def\csname PYG@tok@kp\endcsname{\def\PYG@tc##1{\textcolor[rgb]{0.00,0.44,0.13}{##1}}}
\expandafter\def\csname PYG@tok@w\endcsname{\def\PYG@tc##1{\textcolor[rgb]{0.73,0.73,0.73}{##1}}}
\expandafter\def\csname PYG@tok@kt\endcsname{\def\PYG@tc##1{\textcolor[rgb]{0.56,0.13,0.00}{##1}}}
\expandafter\def\csname PYG@tok@sc\endcsname{\def\PYG@tc##1{\textcolor[rgb]{0.25,0.44,0.63}{##1}}}
\expandafter\def\csname PYG@tok@sb\endcsname{\def\PYG@tc##1{\textcolor[rgb]{0.25,0.44,0.63}{##1}}}
\expandafter\def\csname PYG@tok@k\endcsname{\let\PYG@bf=\textbf\def\PYG@tc##1{\textcolor[rgb]{0.00,0.44,0.13}{##1}}}
\expandafter\def\csname PYG@tok@se\endcsname{\let\PYG@bf=\textbf\def\PYG@tc##1{\textcolor[rgb]{0.25,0.44,0.63}{##1}}}
\expandafter\def\csname PYG@tok@sd\endcsname{\let\PYG@it=\textit\def\PYG@tc##1{\textcolor[rgb]{0.25,0.44,0.63}{##1}}}

\def\PYGZbs{\char`\\}
\def\PYGZus{\char`\_}
\def\PYGZob{\char`\{}
\def\PYGZcb{\char`\}}
\def\PYGZca{\char`\^}
\def\PYGZam{\char`\&}
\def\PYGZlt{\char`\<}
\def\PYGZgt{\char`\>}
\def\PYGZsh{\char`\#}
\def\PYGZpc{\char`\%}
\def\PYGZdl{\char`\$}
\def\PYGZhy{\char`\-}
\def\PYGZsq{\char`\'}
\def\PYGZdq{\char`\"}
\def\PYGZti{\char`\~}
% for compatibility with earlier versions
\def\PYGZat{@}
\def\PYGZlb{[}
\def\PYGZrb{]}
\makeatother

\renewcommand\PYGZsq{\textquotesingle}

\begin{document}

\maketitle
\tableofcontents
\phantomsection\label{index::doc}


Contents:


\chapter{Introduction}
\label{intro:introduction}\label{intro::doc}\label{intro:welcome-to-emg-visualization-project-s-documentation}
The EMG-Visualization-Project is a joined project between the TU Munich faculty
of informatics and faculty of sports and initialized in the scope of an
Interdisciplinary Project (IDP).

The goal of the EMG-Visualization-Project is to give a framework for
Electromyography (EMG) Data visualization. This framework is not intendet for
experts in that you have to now python. The goal of this project is not to
provide an easy to use graphical interface.

The EMG-Visualization-Project utilizes IPython Notebook through which
functionality provided by custom scripts is made available.
This makes it possible to easily set up a workflow to preprocess EMG data.

The following sections will introduce EMG and also some concepts of Machine
Learning.


\section{Electromyography}
\label{intro:electromyography}
An EMG measures electrical current generated in muscles during its contraction.
These measures represent neurmuscular activities.

As to how this electrical signal comes into existence. As a whole the human
body is electically neutral. On the cellular level exist a potential difference
between intra-cellular and extra-cellular fluids, though. If a person decides
to contract a muscle, a stimulus from a neuron is send. This stimulus causes the
cells in the muscle fibre to depolarize as the stimulus propagates along its
surface (i.e. the muscles twitches). This depolarization is accompanied by a
movement of ions which induces an electrical field near each muscle fibre,
which is measured with the EMG electrodes.

The induced field can be measured by electrodes being placed on the skin of a
person (sEMG) or by electrodes being placed in the muscle. The advantages and
disadvantages of these approaches are obvious. The sEMG is easy to administer
but the measured signals are unprecise and noisy, whereas the intramuscular
provides better measurements but is difficult to administer.

The measured signal usually has (before amplification) an amplitude of 0-10mV
and ranges between \(\pm5mV\).

A EMG measurement always consists of multiple Motor Unit Activation Potentials
(MUAPs), that is electrical fields of different muscles. Luckily, MUAPs from
different muscles have different characteristics and therefore a signal can be
decomposed into its constituting MUAPs using mathematical tools like \href{http://en.wikipedia.org/wiki/Wavelet\_transform}{wavelet
transform} or \href{http://en.wikipedia.org/wiki/Fourier\_transform}{fourier
transform}.


\section{Machine Learning}
\label{intro:machine-learning}
TODO: Should put emphasis on how ML is used in the context of this project.


\chapter{Materials and Methods}
\label{materials::doc}\label{materials:materials-and-methods}
This chapter introduces technologies used in the scope of this project. For more
details refer to the respective official documentations.


\section{Pandas}
\label{materials:pandas}
Pandas is a python library for data analysis. Pandas primarly provides simple to
use and efficient data structures and is intended to use in conjunction with
other libraries providing, for example modeling functionality.
todo:: Describe how exactly Pandas is used in context of this project


\section{IPython Notebook}
\label{materials:ipython-notebook}
IPython Notebook as part of IPython providing a rich architecture for
interactive computing. As the name suggests, IPython focuses on Python as
language. The architecture of IPython, however, is designed such that support
of other languages such as Haskell or R is possible. For the remainder of this
document python is meant if referred to code or programming.

IPython Notebook is a web-based, interactive compuational environment and
allows the combination of source code, and other components such as rich text
and rich media.

A notebook consists of different types of executable cells:
* Code cells
* Markdown cells
* Raw cells
* Heading cells

\textbf{Code} cells are the default type and intended to hold program code. There are no
limitations to what can be written inside, everything that would work in a
regular python script works here as well. If a code cell is executed, the
contained program code is evaluated and the result shown below the cell.

\textbf{Markdown Cells} contain text in the markdown format, it is also possible to
include LaTeX with the standard latex \emph{\$\$} notation. If a Markdown cell is
executed, the content is parsed and displayed.

\textbf{Raw Cells} are not executed, or expressing it differently, the output is
exactly the same as the input.

\textbf{Heading Cells} are also rendered to rich text and used to give some
structure to the notebook. There are six different headings.

A notebook is thus a completely new way of ``programming''. Consider a usual
python (or R or octave or whatever) script. One would have a source file
containing all the program code and possibly some comments. Each time a new
part is added, the whole script has to be evaluated which might be time
consuming depending on the task. Descriptions and documentation to what the
program does exist in separate files.

With a notebook it is possible to have all this in one place. To a task or
snipped of code a markdown cell can be written describing what the cell does,
even if the description contains tricky formulas. Cells can be executed
independently of each other and the result of the computation is immediatly
visible.


\chapter{Taxonomy}
\label{taxonomy:taxonomy}\label{taxonomy::doc}
This chapter introduces expressions used in the context of this project. This
is how it is used in the BRML lab. Other institutions and literature might use
different expressions for the same thing, also they most likely mean something
entirely different in other areas.


\section{Hypothesis}
\label{taxonomy:hypothesis}
A (scientific) problem which is to be validated experimentally. It could also be
said a hypothesis is a question for which an answer is sought.


\section{Setup}
\label{taxonomy:setup}
Overhead needed to set up an experiment. This includes the configuration of the
hardware and software, how to record data, how the experiment is to be conducted
and so on.


\section{Experiment}
\label{taxonomy:experiment}
A process resulting in the acceptance or rejection of the hypothesis (answer to
the question asked).

The term experiment as used in this context can be compared to a study in
medicine. It is a process possibly spanning over a longer period of time.
However, there is not just one experiment, but different experiments are
conducted to come to a conclusion.


\section{Modality}
\label{taxonomy:modality}
Can be summarized as a group of sensors. Using an EMG, all the electrodes are a
modality depending on the setup subsets of sensors
might form a modality (e.g. if interested in different muscles).

EMG is often used in conjunction with other systems such as kinematic ones or
Electroencephalography (EEG). In this case each system for itself is a modality.


\section{Data}
\label{taxonomy:data}
Consists of everything used to determine whether or not to accept/reject the
hypothesis.

This includes of course the recorded data of the different modalities and
might inlcude additional data such as time of day, room temperature, gender,
age and so on.


\section{Subject}
\label{taxonomy:subject}
A clearly differentiably entity (person).


\section{Session}
\label{taxonomy:session}
A session is the conduction of an experiment with a certain setup and one
subject over a certain amount of time. Note, that a session might result in
multiple recordings (of the same modality).

Changing the subject, the experiment or the setup does result in a new session.
Changing the time (conducting an experiment with the same person and the same
setup again) does not always result in a new session. If the time span between
the conductions is small (for example drink a glass of water, go to the toilet,
etc) or the conductions are even consecutive, then it is still one session. If
hours, days or even weeks are between two conductions then it is a new session.
This is not well defined and might differ for different occassions.


\section{Recording}
\label{taxonomy:recording}
A stream of continuous data produced during one session. A recording may contain
multiple trials (see below).


\section{Trial}
\label{taxonomy:trial}
The smallest amount of data needed to decide, whether or not to accept or reject
the hypothesis.

For example, to determine if it is possible to identify how a limb moves through
space it would theoretically suffice to record one movement of one person.


\section{Channel}
\label{taxonomy:channel}
Smallest measurable fraction of one modality. In case of an EMG this would be
one electrode.


\section{Sample}
\label{taxonomy:sample}
The term \emph{sample} in this context has to different inflections. A sample can be
understood of with signal processing in mind. In this case a sample is a value
(of a modality) at a specific point of time.

Regarding EMG, the continuous electrical signal generated by muscle activity
is transformed into a continous one by measuring the electrical current in
specific time intervals.

The other interpretation has a statistical background. In this context a
sample is a collection of data drawn from a population. Exemplary, let our
population be all people suffering under Spinal Muscular Atrophy (SMA). Then
our sample would be EMG data of a subset of these people. Given that data, we
want to find a pattern, describing the action the person would like to perform
with its arm, which generalizes to the whole population.

So a sample in the statistical sense would be the recorded data.

In the scope of this poject the term sample will be used in the sense of signal
processing.


\chapter{Implementation}
\label{implementation:implementation}\label{implementation::doc}
This chapter explains the design of the solution, how different components
interact and what functionality is realized with them.

The solution comprises three major parts, going to be illustrated in detail
in following sections.

The first part is about representing an Experiment according to the
\DUspan{xref,std,std-ref}{chap-taxonomy}, the second part is about performing online learning
over the network using distributed data-sources and the third part deals
with streamlining operations on data.


\section{Experiment Representation}
\label{implementation:experiment-representation}
Python is a powerful programming language providing a wide range of
libraries for data analysis and visualization. In particular the
\href{http://pandas.pydata.org/}{Pandas} library provides functionality for
analyzing time series data.

However, organizing all necessary information such as class labels, different
modalities, sessions, recordings and trials is not that convenient with pandas,
scipy and numpy alone.

The classes implemented in package \code{model.model} wrap functionality around
pandas and numpy to alleviate working with that specific data.


\subsection{Defining an Experiment}
\label{implementation:defining-an-experiment}
The idea is to model an experiment similar to a knowledge base in prolog.
This knowledge base contains all information about trials in different
recordings or specific events within the trials and can be quried for those.

Things that are contained in the knowledge base are for example:
\begin{itemize}
\item {} 
The recording

\item {} 
Start and duration of trials

\item {} 
Start and duration of specific events

\item {} 
Labels for trials

\item {} 
Modalities and channels used

\item {} 
Participating subjects

\end{itemize}


\subsection{Managing Trials, Recordings and Events}
\label{implementation:managing-trials-recordings-and-events}
Recordings are the only entities that hold a larger amount of data. During
definition, a recording object may be passed a DataFrame, Array or path to
a file containing the recording's data.

Trials are chunks of the recording. Each trial object holds a reference to
the recording it belongs to. When samples of a trial are requested, the
recording is sliced according to the start point and duration of the trial.

Similar, Events are defined for trials and specify a start point relative to
the start of the trial and optionally a duration.

The chunk of the recording is returned as an instance of class
\code{DataContainer}. This object then contains all necessary information such as
the events defined for the trial, sampling rate, names of channels and so on.


\subsection{Summary}
\label{implementation:summary}
Representation of an experiment is a sort of knowledge base holding all
relevant information. It acts as a wrapper around pandas and numpy to allow
better handling of data in this setting.


\section{Streamlining Data Preparation}
\label{implementation:streamlining-data-preparation}
The primary goal of this part is chaining of operations used for
data preparation.

The functionality is implemented using the \DUspan{xref,std,std-ref}{subsec-decorator-pattern}.
This design choice is motivated by allowing better implementation of more
complex functionality, a strong guideline how to implement new functions and
an elegant way of chaining different functions.


\subsection{Decorator Pattern}
\label{implementation:subsec-decorator-pattern}\label{implementation:decorator-pattern}
This design pattern wraps an object and augments it with additional functionality
independently from other instances of the same class.

It allows division of functionality between classes, where each class has
a certain focus. Thus it is well suited in this case. In this context, each
class implements a certain data preparation functionality.


\subsection{Generator and Lists}
\label{implementation:generator-and-lists}
Each decorator class can act as a generator or simply returning a list of
\code{DataContainer}.

A generator in python is function that acts like an interator. This means,
the function returns elements lazily, that is on demand.

The \code{WindowDecorator} is a perfect example of this. When creating windows
from a sequence, new arrays are created. If windows are generated from multiple
sequences at once, a lot of memory might be used.

When using a generator, a new windows is created when explicitly asked for
leading to reduced memory usage.


\subsection{Summary}
\label{implementation:id1}
By using the decorator pattern one class implements a specific function
related to data preparation. Different decorators can be stacked to chain
functions.

Decorators can act as generators, thus lowering the resource consumption.


\section{Online Learning}
\label{implementation:online-learning}
This part explains functionality of working with distributed datasources and
sending it over the network.


\subsection{Nanomsg and ZeroMQ}
\label{implementation:nanomsg-and-zeromq}
Nanomsg and ZeroMQ are both high performance, asynchroneous messaging libraries.
In fact, Nanomsg is the successor of ZeroMQ and developed by the same people.

Nanomsg and ZeroMQ abstract networking and allow easy implementation of (large)
distributed systems.

Both libraries implement a range of message patterns, such as the publisher-
subscriber pattern, being used in this implementation.

Nanomsg is newer and fixes some of the shortcomings of ZeroMQ. The support and
documentation of Nanomsg is not yet as good as the one for ZeroMQ, though.
Especially when it comes to working on windows, ZeroMQ is very convenient.

Both libraries are supported.


\subsection{From Datasource to Learning Scheme}
\label{implementation:from-datasource-to-learning-scheme}
To work with distributed data the publisher-subscriber pattern is used. In this
pattern there is one information broker. This broker broadcasts messages to a
specific topic. After subscribing to a topic, the subscriber receives messages
from the respective publisher.

In this solution, the messages of the publisher are a specific modality. To
collect the data, \code{DataSource} objects are used. They retrieve data from
a real-time system and put them into a queue. The publisher grabs these
messages and publishs them into the network.

Somewehere else a subscriber listens for messages for his modality. If the
subscriber gets a message, the message is deserialized and put into a queue.

From there it can be fed into an online learning scheme.


\chapter{Indices and tables}
\label{index:indices-and-tables}\begin{itemize}
\item {} 
\DUspan{xref,std,std-ref}{genindex}

\item {} 
\DUspan{xref,std,std-ref}{modindex}

\item {} 
\DUspan{xref,std,std-ref}{search}

\end{itemize}



\renewcommand{\indexname}{Index}
\printindex
\end{document}
